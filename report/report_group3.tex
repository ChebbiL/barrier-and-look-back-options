\documentclass[11pt,a4paper,fleqn]{article}

\usepackage{multirow}
\usepackage[table,xcdraw]{xcolor}
\usepackage{amssymb,latexsym, amsmath}
\usepackage{amscd}
\usepackage{xcolor}
\definecolor{dkgreen}{rgb}{0,0.6,0}
\definecolor{gray}{rgb}{0.5,0.5,0.5}
\definecolor{mauve}{rgb}{0.58,0,0.82}
\usepackage{amssymb}
\usepackage{amsthm}
\usepackage{mathdots}
\usepackage{amsmath,amsfonts,latexsym}
\usepackage{graphicx}
\usepackage{amsmath}
\usepackage{amsfonts}
\usepackage{listings}
\usepackage{color}

\usepackage{subcaption}
\usepackage{dsfont}
\usepackage{placeins}

\usepackage{pdfpages}
\usepackage{hyperref}

\usepackage[top=1in, left=1in, right=1in, bottom=1in]{geometry}  % This line determines the page layout.
% Consult the documentation of the geometry package if you want to change it.

% include these lines if you want to typeset your report in a sans serif font
%\renewcommand{\sfdefault}{cmbr}
%\renewcommand{\ttdefault}{cmtl}
%\renewcommand{\familydefault}{\sfdefault}

\newtheorem{remark}{Remark}
\newtheorem{Cor}{Corollary}
\newtheorem{definition}{Definition}
\newtheorem{proposition}{Proposition}
\newtheorem{theorem}{Theorem}
\newtheorem{property}{Property}
\newtheorem{lemma}{Lemma}
\newtheorem{question}{Question}
\newtheorem{example}{Example}
\newcommand\tr{\mathrm{tr}}
\newcommand\Tr{\mathrm{Tr}}
\newcommand\St{\mathrm{St}}
\newcommand\ad{\mathrm{ad}}
\newcommand\Ad{\mathrm{Ad}}
\newcommand\goth{\mathfrak}
\newcommand\R{\mathbb R}
\newcommand\Z{\mathbb Z}
\newcommand\ind{\mathrm{ind}\,}
\newcommand\cork{\mathrm{corank }\,}
\newcommand\Ker{\mathrm{Ker}\, }
\newcommand\Ann{\mathrm{Ann}\, }
\newcommand\ann{\mathrm{ann}\, }
\newcommand\spann{\mathrm{span} }
\newcommand\rank{\mathrm{rank}\, }
\newcommand\corank{\mathrm{corank}\, }
\newcommand\codim{\mathrm{codim}\, }
\newcommand\Sing{{\mathsf{Sing}}}
%\newcommand\Inv{\mathsf{Inv}_{\mathrm{formal}}(a)}
\newcommand\Inv{Y_{\mathrm{formal}}(\goth g, a)}
\newcommand\kfield{{\mathbb{K}}}
%\newcommand\deg{\mathrm{deg}\, }
\newcommand\trdeg{\mathrm{tr.deg.}\,}
\newcommand\rk{\mathrm{rk}\, }
\newcommand\grad{\mathrm{grad}\, }

\newcommand{\svert}{{s_{\mathrm{vert}}}}

\newcommand{\shor}{{s_{\mathrm{hor}}}}

\newcommand{\sjord}{{s_{\mathrm{Jord}}}}

\newcommand{\Lvert}{L_{\mathrm{vert}}}

\newcommand{\Lhor}{L_{\mathrm{hor}}}

\newcommand{\dimO}{\dim \mathcal O_{\mathrm{reg}}}

\newcommand{\codimO}{\codim \mathcal O_{\mathrm{reg}}}

\newcommand{\dimSt}{\dim \mathrm{St}_{\mathrm{reg}}}

\pagestyle{headings}

\author{{\LARGE Simulation Methods for Finance}\\under the supervision of Professor Harry Zheng
\\[5cm] Authors:\\
%Tresnia Berah\\
Thomas Espel\\ Konstantin Kulak\\ Callum MacIver\\Zhentian Qiu\\Vera Zhang\\[6.5cm]}
\title{\vspace*{-2cm}\begin{flushleft}\includegraphics[width=7cm]{graphs/imperial.png}\end{flushleft}
\vspace*{4cm}
\Huge\sffamily Simulation Methods for Barrier and Look-back Options}
\date{\sffamily\today}
\pagenumbering{Roman}
\begin{document}
\maketitle
\thispagestyle{empty}

\newpage
\tableofcontents
\newpage
\pagenumbering{arabic}
\setcounter{page}{1}

\part{Basic Task}
\section{Generate Random Number}
\subsection{Presentation}
There are two methods we used to generate random numbers.
The first one is to generate a uniform distribution and then transform it into Standard Normal Distribution. The second method we tried is the system build-in function \texttt{random}, which can directly generate a variable following Standard Normal Distribution.
\subsection{Generating the uniform distribution}
To generate a uniform distribution we tried two ways: one can use the system build-in methods \texttt{rand}. The function will return a number between 1 and $2^{15} -1$ pseudo-randomly. The range is around 30,000, way below 100,000, the sample size we plan to generate. In other words, when we use \texttt{rand} to simulate 100,000 entries, there will be numbers appear more than once, which will generate correlations. We prefer the second way to generate a uniform distriubtion: linear congruential generator.

$$n_i = (an_{i-1}) \ mod\ m$$for $i=1,2,...,10,000$, and we let $ a = 7^5$, and $m = 2^{31}-1$, which gives about 2 billion points. We run 100,000 times and will only use $0.005\%$ of all points. Theoretically, no pattern should appear. ${n_i \over m}$ will give a distribution close to uniform distribution from (0,1).

\subsection{Generating the normal distribution}
To transform the uniform distribution sample into Standard Normal Distribution sample, we have tried three methods. By Central Limit theory,

$$ Z_n = \frac{\sum_{i=1}^{n}X_i-n\mu}{\sqrt{n}\sigma}$$where $X_i$ is from the uniform distribution we generated previously. $Z_n$ converges in distribution to Standard Normal Distribution, however, it requires n, the number of uniform distribution, to be sufficiently large. Therefore this method requires significantly large number of simulations and speed is decreased consequently.

We also tried Box-Muller methods  \cite{lectures}

$$Z_1 = \sqrt{-2lnX_1}sin(2\pi X_2),\  \Z_2=\sqrt{-2lnX_1}cos(2\pi X_2)$$Since its simulation involves computation of \textit{sine} and \textit{cosine}, the speed is slow.
We prefer the last method, Marsaglia Polar method.

$$ let \ V_1 = 2U_1-1,\  V_1 = 2U_2-1.$$
where $U_1$ and $U_2$ are two independent uniform distribution.
 Let\ $W = V_1^2+V_2^2.$\ If $W>1$, return to the beginning. Otherwise,

$$N_1 = \sqrt{\frac{(-2logW)}{W}}V_1,\ N_2 = \sqrt{\frac{(-2logW)}{W}}V_2$$As the computation does not involve \textit{sine} and \textit{cosine}, it is generally faster than Box-Muller.



\begin{table} [h!]
\centering
\begin{tabular}{| c| c| c| c| }
\hline
\textbf{ Method} &\textbf{ Mean} & \textbf{Variance}&\textbf{ Time (seconds)} \\ \hline
\textbf{\texttt{ rand}  + CLT} & 1.77 e-5 & 0.999909 & 90.34 \\  \hline
\textbf{\texttt{ rand}  + Box-Muller} & -6.80 e-5 & 1.000703&8.64\\ \hline
\textbf{\texttt{ rand}  + Marsaglia Polar} & 1.70 e-5& 1.000472& 6.40\\ \hline
\textbf{ LCG + CLT} & 1.52 e-5& 0.999973&236.3\\ \hline
\textbf{LCG + Box-Muller} & -1.27 e-4 & 0.999797&11.59\\ \hline
\textbf{ LCG + Marsaglia Polar} & 7.36 e-5 & 0.999979&10.52\\ \hline
\textbf{\texttt{random} }& 3.58 e-4 & 1.000210&14.98\\ \hline
\end{tabular}
\caption{We note that the Marsaglia method is faster compared to the other methods. Note that the run time depends on the computer used for the simulation (here a 3.4 GHz Intel Core~i7). We have generated 100,000,000 variables to obtain these results.}
\end{table}

Table 1 is the simulation results of Standard Normal distribution and time used by different methods for the generation of 100,000,000 random variables. As shown in the table, when using CLT\footnote{Central Limit Theorem} to generate standard normal distribution, the time needed is significantly longer than the rest methods. Marsaglia Polar method gives a variance closer to 1 compared to Box-Muller methods. Linear congruential generator gives a variance closer to 1 compared to \texttt{rand} method. Hence we use the combination of linear congruential generator and Marsaglia methods to simulate random variables. This was also the opportunity for us to have full control over the generation of random numbers, as it is a critical step towards efficient simulations \cite{lectures}. It also features very good results in terms of statistical parameters and time.


\FloatBarrier

\section{European Option}

\subsection{Presentation}
The asset price in a risk neutral probability space $(\Omega, \mathcal{F}, (\mathcal{F}_t)_{0\leq t\leq T}, P)$ follows Geometric Brownian Motion,

$$dS_t=rS_tdt+\sigma S_tdW_t, \ 0\leq t \leq T$$
with initial price $S_0 = S$, where $r$ is riskless interest rate, $\sigma$ volatility, and $W_t$ the standard Brownian motion. A European call option price at time $t$ with maturity time $T$ is given by
$$C_t = E[e^{-r(T-t)}(S_T-K)^+|\textit{F}_t]$$

For the basic task, we let $S_0 = 100$, $K=100$, interest rate $r = 0.05$, volatility $\sigma = 0.4$, maturity time $T =1$ and initial time $t=0$. We use Monte Carlo method to simulate

$$S_T=S_0e^{(r-\frac{1}{2}\sigma ^2)T+\sigma \sqrt{T}Z} $$

and get sample distribution of $(S_T-K)^+$. By Black-Scholes formula,


$$C_{bs}(S_0, K) = N(d_1)S_0 - N(d_2)Ke^{-rT}$$

where

$$d_1(S_0,K)=\frac{1}{\sigma \sqrt{T}}[ln(\frac{S_0}{K})+(r+\frac{\sigma^2}{2})T],\ d_2(S_0,K) = d_1-\sigma \sqrt{T}$$

The closed-form Greeks are calculated the following way:
$$\delta_{bs} = \frac{\partial C}{\partial S_0}=\phi(d_1),\hspace*{.5cm} \Gamma_{bs} = \frac{\partial^2 C}{\partial S_0^2}=\frac{\Phi'(d_1)}{S_0\sigma \sqrt{T}},\hspace*{.5cm} \nu_{bs} =  \frac{\partial C}{\partial \sigma}=\Phi'(d_1)\sqrt{T}$$

\subsection{Likelihood Ratio Greeks}
To calculate the Greeks from simulation, we compare likelihood ratio method and Pathwise method. The Call option price is given by

$$C = e^{-rT}E[(S_T-K)^+]=e^{-rT}\int(S_T-K)^+ h_{S_0}(S_T) dS_T$$

where $ h_{S_0}(S_T)$ is the probability density function of $(S_T-K)^+$. Then by Likelihood ratio method, the partial derivative of C with respect to $S_0$ is

$$\frac{\partial C}{\partial S_0}=\int(S_T-K)^+\frac{d}{dS_0}h_{S_0}(S_T) dS_T=E[(S_T-K)^+\frac{h'_{S_0}(S_T)}{h_{S_0}(S_T)}] $$

And the lognormal density function of $S_T$ is given by
$$h(x)=\frac{1}{x\sigma \sqrt{T}}\phi(\xi(x)), \ \ \xi(x)=\frac{ln(x/S_0)-(r-\frac{1}{2}\sigma^2)T}{\sigma \sqrt{T}}  $$

Therefore \cite{lectures},

$$Delta_{LR}=\frac{\partial C}{\partial S_0}=E[e^{-rT}(S_T-K)^+\frac{Z}{S_0\sigma\sqrt{T}}]$$
where $Z~N(0,1).$

$$Gamma_{LR LR} = \frac{\partial^2 C}{\partial  S_0^2 }=E[e^{-rT}(S_T-K)^+(\frac{Z^2-1}{S_0^2\sigma^2 T } - \frac{Z}{S_0^2 \sigma \sqrt{T}})]$$

$$Vega_{LR} = \frac{\partial C}{\partial \sigma}=E[e^{-rT}(S_T-K)^+\left(\frac{Z^2-1}{\sigma}-Z\sqrt{T}\right)]$$

\subsection{Pathwise Derivative Greeks}
By pathwise methods, the Greeks are given as following\footnote{For gamma, there are two different methods.} \cite{lectures}, with $Z\sim N(0,1)$:

$$Delta_{PW}=\frac{\partial C}{\partial S_0}=E[e^{-rT}\mathds{1}_{S_T>K}\frac{S_T}{S_0}]$$

$$Gamma_{LRPW} = \frac{\partial^2 C}{\partial  S_0^2 }=E[e^{-rT}\mathds{1}_{S_T>K}\frac{KZ}{S_0^2\sigma\sqrt{T}}]$$

$$Gamma_{PWLR} = \frac{\partial^2 C}{\partial  S_0^2 }=E[e^{-rT}\mathds{1}_{S_T>K}\frac{S_T}{S_0^2}\left(\frac{Z}{\sigma\sqrt{T}}-1\right)]$$

$$Vega_{PW} = \frac{\partial C}{\partial \sigma}=E[e^{-rT}\mathds{1}_{S_T>K}S_T(\sqrt{T}Z-\sigma T)]$$

The results calculated from the closed-form formulae and the simulations are represented in Table \ref{tab:euroresults}.


\begin{table}[h!]
\centering
\begin{tabular}{|c|c|c|c|c|c|}
\hline
\multicolumn{2}{|c|}{\multirow{2}{*}{}}          & \multirow{2}{*}{\textbf{Black -Scholes}} & \multicolumn{3}{c|}{\textbf{Monte Carlo Simulation}} \\ \cline{4-6}
\multicolumn{2}{|c|}{}                           &                                          & \textbf{1,000}  & \textbf{10,000} & \textbf{100,000} \\ \hline
\multicolumn{2}{|c|}{\textbf{Option Price}}      & 18.023                                   & 19.0162         & 17.6732         & 18.0014          \\ \hline
\multirow{2}{*}{\textbf{Delta}} & \textbf{LR}    & \multirow{2}{*}{0.627409}                & 0.662561        & 0.612362        & 0.627889         \\ \cline{2-2} \cline{4-6}
                                & \textbf{PW}    &                                          & 0.64485         & 0.623239        & 0.627453         \\ \hline
\multirow{3}{*}{\textbf{Gamma}} & \textbf{PW LR} & \multirow{3}{*}{0.0094605}               & 0.0101213       & 0.00919597      & 0.00945029       \\ \cline{2-2} \cline{4-6}
                                & \textbf{LR PW} &                                          & 0.00994418      & 0.00930474      & 0.00944593       \\ \cline{2-2} \cline{4-6}
                                & \textbf{LR LR} &                                          & 0.00975465      & 0.00918059      & 0.0095502        \\ \hline
\multirow{2}{*}{\textbf{Vega}}  & \textbf{LR}    & \multirow{2}{*}{37.842}                  & 39.0186         & 36.7224         & 38.2008          \\ \cline{2-2} \cline{4-6}
                                & \textbf{PW}    &                                          & 40.4851         & 36.7839         & 37.8012          \\ \hline
\end{tabular}
\caption{\label{tab:euroresults}Table with the result of our simulations for different Monte-Carlo iterations. We can clearly see the improvement between 1,000 and 10,000 simulations.}
\end{table}

We can see that the simulation results improve and get closer to the theoretical values as the simulation number increases. Hence, to compare the results from different methods, we will only analyze the simulation results from the largest simulation number - 100,000 in this case. This is even more clear on the graphs (Figure \ref{fig:eurographs}).

\begin{figure}
  \centering
      \begin{subfigure}[b]{0.45\textwidth}
          \includegraphics[width=\textwidth]{graphs/eurodeltalr.png}
          \caption{Delta with LR method}
      \end{subfigure}
      \begin{subfigure}[b]{0.45\textwidth}
          \includegraphics[width=\textwidth]{graphs/eurodeltapw.png}
          \caption{Delta with PW method}
      \end{subfigure}

      \begin{subfigure}[b]{0.3\textwidth}
          \includegraphics[width=\textwidth]{graphs/eurogammalrlr.png}
          \caption{Gamma with LR-LR method}
      \end{subfigure}
      \begin{subfigure}[b]{0.3\textwidth}
          \includegraphics[width=\textwidth]{graphs/eurogammalrpw.png}
          \caption{Gamma with LR-PW method}
      \end{subfigure}
      \begin{subfigure}[b]{0.3\textwidth}
          \includegraphics[width=\textwidth]{graphs/eurogammapwlr.png}
          \caption{Gamma with PW-LR method}
      \end{subfigure}

      \begin{subfigure}[b]{0.45\textwidth}
          \includegraphics[width=\textwidth]{graphs/eurovegalr.png}
          \caption{Vega with LR method}
      \end{subfigure}
      \begin{subfigure}[b]{0.45\textwidth}
          \includegraphics[width=\textwidth]{graphs/eurovegapw.png}
          \caption{Vega with PW method}
      \end{subfigure}

      \caption{\label{fig:eurographs}A graphical analysis of the errors of the Greeks depending on the number of simulations. We notice that there is a significant increase above 1000 simulations in all cases. This is very clear when looking at vega which tends to have a very high error variance below 1,000 simulations. In order to generate these graphs, we have done 1,000 simulations for each value, at every power of 10.}
\end{figure}



\begin{table} [h!]
  \centering
\label{eco:values}
\begin{tabular}{|l|c|c|c|}
\hline
1,000,000 MC simulations      & \textbf{Error Mean} & \textbf{Error Variance} & \textbf{Time (seconds)} \\ \hline
\textbf{Option Price} & 7.80 e-2 & 3.52 e-3 & 2.79 e-2 \\ \hline
\textbf{Delta LR} & 4.1 e-3 & 9.7 e-6 & 1.4 e-3\\
\textbf{Delta PW} & 1.8 e-3 & 1.8 e-6 & 9.9 e-4\\ \hline
\textbf{Gamma PWLR} & 6.1 e-5 & 2.0 e-9 & 1.4 e-3\\
\textbf{Gamma LRPW} & 3.5 e-5 & 7.4 e-10 & 1.4 e-3\\
\textbf{Gamma LRLR} & 1.9 e-4 & 2.1 e-8 & 4.7 e-3\\ \hline
\textbf{Vega LR} & 7.7 e-1 & 3.3 e-1 & 4.6 e-3\\
\textbf{Vega PW} & 2.4 e-1 & 3.2 e-2 & 1.6 e-3\\ \hline
\end{tabular}

\caption{Sample from our simlation dataset with 100,000 Monte-Carlo simlations. Note that the run time depends on the computer (here a 3.1 GHz Intel Core~i5). We computed the absolute error and since the option price is computed using the closed form formula no error is represented.}

\end{table}

We now have a closer look at 100,000 simulations, which is the industry standard and which is above the 1,000 simulations threshold we have noticed on the graphs.
For delta, although the fastest methods is LR, it has a significantly lower accuracy. Gamma LRLR is the most accurate but it is the slowest, so gamma PWLR seems to be a good compromise. Vega PW is both faster and more accurate than vega LR.

For the following section, PW methods will be infeasible to do for Barrier Option. Hence, we will use likelihood ratio method when calculating the Greeks.

\FloatBarrier
\newpage
\part{Main Task}
\section{Barrier Option}
\subsection{Presentation}
Let $T$ denote option expiration time and $[0,T]$ lookback period. For $T_0 \leq t\leq T$ denote by

$$m^T_{0} =\underset{0 \leq t \leq T}{min}  S_t, \ \   M^t_{0} = \underset{0 \leq t \leq T}{max} S_t$$

For an up-and-out barrier call option $A_T=(S_T-K)^+\mathds{1}_{max 0\leq t\leq T \  S_t\leq B}$, where $B$ is a barrier leve and $\mathds{1}_S$ is an indicator function.

We simulate the path of $S_t$ by dividing the maturity time $T$ into 1,000 steps, and we simulate 5,000 such paths. We have an $M^T_0$ for each path. However, this method is bordensome as we need to generate each step for each path. The simulation process is long.

\subsection{Another Method using Rayleigh Distribution}
Taking into consideration the arguments from above, we tried second method using Rayleigh Distribution to simulate the distribution of $M^t_0$ directly. The maximum of a standard Brownian motion starting at the origin to be at b at time 1 over period [0, T] has the Rayleigh distribtuion \cite{barrieropt, glasserman, exooptions}

$$F(x) = 1 - e^{-2x(x-b)}, \ x \geq b.$$

solving the equation $F(x) = u, u\in (0,1)$ has roots
$$x = {b \over 2} \pm {\sqrt{b^2-2log(1-u)}\over 2}$$

Hence, at time T with $S_T$,
$$M^T={S_T+\sqrt{S_T^2-2TlogU}\over 2}$$
And we simulate $S_T$ by Black-Scholes formula
$$S_T=S_0e^{(r-{1\over 2}\sigma^2)T+\sigma \sqrt{T}Z}$$

Therefore,
$$M^T=S_0e^{{1\over2}log({S_T\over S_0})+\sqrt{log({S_T\over S_0})^2-2\sigma^2TlogU}}$$

With this method, we only need to generate $S_T$ for each path and one random variable for the uniform distribution to get the $M^T$. It reduces the amount of simulation by 500 times (as before, 1,000 $S_t$ need to be generated for each path).

As we mentioned previously, it is impossible to find the partial direvative of the indicator function $\mathds{1}_{M^T_0\leq B}$  with respect to $S_0$. Hence pathwise method is eliminated by us. For the likelihood ratio method, we find the differentiation of joint cumulative density function of $S_T$ and the  $M_T $ to be
$$f_{uo}(x, m, T)={1\over \sqrt{T}}\left(\phi({x-\mu T \over \sqrt{T}})-e^{2m\mu} \phi({x-2m-\mu T\over \sqrt{T}})\right) $$

where $x = {1\over \sigma}ln {S_T\over S_0}$, $\mu=\frac{1}{\sigma}(r-\frac{\sigma^2}{2})$ and $m = {1\over \sigma}ln {M_T\over S_0}$. Then we can find the Greeks accordingly, the details of the computation are in Appendix \ref{sec:bocomp}.\\


\begin{table}
\centering
\begin{tabular}{|c|c|c|}
\hline
     & \textbf{Closed Form} & \textbf{Monte Carlo 100,000 Simulation}\\ \hline
\textbf{Option Price} & 15.18 & 15.11 \\ \hline
\textbf{Delta LR} & 0.776 & 0.769\\ \hline
\textbf{Gamma LRLR} & 2.58 e-3 & 2.89 e-3\\ \hline
\textbf{Vega LR} &15.55 &16.02\\ \hline
\end{tabular}
\caption{Theoretical results compared to simulation results with the same parameters as the European option and a barrier at 100.}
\end{table}

\begin{table}
\centering
\begin{subtable}{\textwidth}
  \centering
\begin{tabular}{|l|c|c|c|}
\hline
100,000 MC simulations      & \textbf{Error Mean} & \textbf{Error Variance} & \textbf{Time (seconds)} \\ \hline
\textbf{Delta LR} & 1.508 e-2 & 1.40 e-4 & 2.109\\ \hline
\textbf{Gamma LRLR} & 6.186 e-4 & 2.131 e-7& 2.118\\ \hline
\textbf{Vega LR} & 2.405 & 3.435 & 2.113\\ \hline
\end{tabular}
\caption{Error statistics and computation time for the Newton-Raphson method. We computed the absolute error and since the option price is computed using the closed form formula no error is represented.}
\end{subtable}\\



\vspace*{.5cm}
\begin{subtable}{\textwidth}
  \centering
\begin{tabular}{|l|c|c|c|}
\hline
1,000,000 MC simulations      & \textbf{Error Mean} & \textbf{Error Variance} & \textbf{Time (seconds)} \\ \hline
\textbf{Option Price} & 7.58 e-2 & 3.24 e-3 & 6.27 e-2 \\ \hline
\textbf{Delta LR} & 1.036 e-2 & 3.029 e-5 & 9.284 e-2\\ \hline
\textbf{Gamma LRLR} & 1.895 e-4 & 1.895 e-8& 1.475 e-1\\ \hline
\textbf{Vega LR} & 7.507 e-1 & 3.220 e-1 & 1.097 e-1\\ \hline
\end{tabular}
\caption{Error statistics and computation time for the Rayleigh method. We computed the absolute error and since the option price is computed using the closed form formula no error is represented.}
\end{subtable}
\caption{Sample from our simulation dataset with the new fast method for barrier option simulation. It is clear that the results have significantly improved compared with the previous method. Note that the run time depends on the computer used (here a 3.4 GHz Intel Core i7). However, due to the extremely slow computation time for LR methods, we only computed up to 100,000 samples compared to 1,000,000 for the Rayleigh method.}
\end{table}

To analyze further the accuracy of our simulation, we plot the simulated option prices and Greeks versus their theoretical values as barrier increases (Figure \ref{fig:barriergraphs}).

\begin{figure}[h!]
  \centering
      \begin{subfigure}[b]{0.45\textwidth}
          \includegraphics[width=\textwidth]{graphs/barrierolddeltalrtime.png}
          \caption{Delta with LR method}
      \end{subfigure}
      \begin{subfigure}[b]{0.45\textwidth}
          \includegraphics[width=\textwidth]{graphs/barrierdeltalrtime.png}
          \caption{Delta with Rayleigh method}
      \end{subfigure}

      \begin{subfigure}[b]{0.45\textwidth}
          \includegraphics[width=\textwidth]{graphs/barrieroldgammalrlrtime.png}
          \caption{Gamma with LRLR method}
      \end{subfigure}
      \begin{subfigure}[b]{0.45\textwidth}
          \includegraphics[width=\textwidth]{graphs/barriergammalrlrtime.png}
          \caption{Gamma with Rayleigh method}
      \end{subfigure}

      \begin{subfigure}[b]{0.45\textwidth}
          \includegraphics[width=\textwidth]{graphs/barrieroldvegalrtime.png}
          \caption{Vega with LR method}
      \end{subfigure}
      \begin{subfigure}[b]{0.45\textwidth}
          \includegraphics[width=\textwidth]{graphs/barriervegalrtime.png}
          \caption{Vega with Rayleigh method}
      \end{subfigure}

      \caption{\label{fig:barriergraphs}A graphical analysis of the error of the Greeks depending on the computation time. We notice that the method with Rayleigh distribution is between 100 and 1,000 times faster. In order to generate these graphs, we have done 1,000 simulations for each value. However, due to the extremely slow computation time for LR methods, we only computed up to 100,000 samples compared to 1,000,000 for the Rayleigh method.}
\end{figure}
\FloatBarrier

\section{Look-back Option}
\subsection{Presentation}
Look-back call option with fixed strike price $K$ has payoff $(M^T_{0}-K)^+$. The call option price at time $t$ is

$$c(S_0,K,t) = e^{-r(T-t)}E[(max(M^t_0,M^T_t)-K)^+|\mathcal{F}_t] $$

The closed-form formula for fixed strike look-back call option at time 0 (see \cite{lectures}, \cite{exooptions}) is

$$c(S_0,K,0)=C_{bs}(S_0,K) + \frac{S_0\sigma^2}{2r}\{ \Phi[d_2(S_0,K)]-e^{-rT}\frac{S_0}{K}^{-\frac{2r}{\sigma^2}} \Phi[-d_1(K,S_0)]\}$$


The probability density function of the distribution of maximum $S_t$ for standard Brownian motion for the period of (0, T) is
$$f_{(m)} = \frac{2}{\sqrt{T}}\phi\left(\frac{m-aT}{\sqrt{T}}\right)-2ae^{2am}\Phi\left(\frac{-m-aT}{\sqrt{T}}\right)$$

and the cumulative density function is
$$F_{(M_T>m)} = \Phi\left(\frac{m-aT}{\sqrt{T}}\right)-e^{2am}\Phi\left(\frac{-m-aT}{\sqrt{T}}\right)$$

We used Newton-Raphson method to solve for the above cumulative density function $F=U,$ the uniform distribution. Hence, for each random number we generate from [0,1], we receive one m by solving for F. To get $M^T$ of stock price which follows geometric Browninan motion with starting price $S_0$, we let $M^T=S_0e^{\sigma m }$.\\


\subsection{Another Method using Rayleigh Distribution}
 In a similar way as for the barrier option, we have also implemented a faster and more efficient method. The results can be compared on the plots.
\begin{table}[h!]
  \centering
\begin{tabular}{|c|c|c|c|c|c|}
\hline
\multicolumn{2}{|l|}{100,000 MC Simulations}& \textbf{Closed Form} & \textbf{Monte-Carlo 100,000 Simulation} \\ \hline
\multicolumn{2}{|c|}{\textbf{Option Price}}      & 37.76 &  37.715       \\ \hline
\multirow{2}{*}{\textbf{Delta}} & \textbf{LR}    & \multirow{2}{*}{1.329}                &1.329      \\
                                & \textbf{PW}    &                                          & 1.322              \\ \hline
\multirow{3}{*}{\textbf{Gamma}} & \textbf{PW LR} & \multirow{3}{*}{0.021}                 & 0.021       \\
                                & \textbf{LR PW} &                                           & -0.0023      \\
                                & \textbf{LR LR} &                                          & 0.021        \\ \hline
\multirow{2}{*}{\textbf{Vega}}  & \textbf{LR}    & \multirow{2}{*}{98.68} &    100.56      \\
                                & \textbf{PW}    &                                             &      111.02     \\ \hline
\end{tabular}
\caption{Theoretical results compared to simulation results with the same parameters as the European option. Note in this specific case we have a problem with Vega PW.}
\end{table}
\begin{table}
\centering
\begin{subtable}{\textwidth}
  \centering
\begin{tabular}{|l|c|c|c|}
\hline
100,000 MC simulations      & \textbf{Error Mean} & \textbf{Error Variance} & \textbf{Time (seconds)} \\ \hline
\textbf{Delta LR} & 5.568 e-3 & 1.768 e-5 & 0.269\\ \hline
\textbf{Gamma PWLR} & 8.383 e-5& 3.638 e-9& 0.261\\ \hline
\end{tabular}
\caption{Error statistics and computation time for the Newton-Raphson method. We computed the absolute error and since the option price is computed using the closed form formula no error is represented.}
\end{subtable}\\

\vspace*{.5cm}
\begin{subtable}{\textwidth}
  \centering
\begin{tabular}{|c|c|c|c|c|c|}
\hline
1,000,000 MC simulations      & \textbf{Error Mean} & \textbf{Error Variance} & \textbf{Time (seconds)} \\ \hline
\textbf{Price} & 9.03 e-2 & 4.71 e-3  & 0.126  \\ \hline
\textbf{Delta LR} & 5.44 e-3 & 1.67 e-5 & 0.145\\ \hline
\textbf{Delta PW} & 9.34 e-4 & 5.05 e-7 & 0.068\\ \hline
\textbf{Gamma LRLR} & 2.80 e-4 & 4.39 e-8& 0.153\\ \hline
\textbf{Gamma PWLR} & 2.91 e-4 & 4.93 e-8& 0.153\\ \hline
\textbf{Vega LR} & 1.189 & 0.779 & 0.157\\ \hline
\textbf{Vega PW} &7.503 & 0.050 & 0.061\\ \hline
\end{tabular}
\caption{Error statistics and computation time for the Rayleigh method. We computed the absolute error and since the option price is computed using the closed form formula no error is represented.}
\end{subtable}
\caption{Sample from our simulation dataset with the new fast method for barrier option simulation. It is clear that the results have significantly improved compared to the previous method. Note that the run time depends on the computer used (here a 3.4 GHz Intel Core i7).}
\end{table}

 \begin{figure}[h!]
   \centering
      \begin{subfigure}[b]{0.45\textwidth}
           \includegraphics[width=\textwidth]{graphs/lookbackolddeltalrtime.png}
           \caption{Delta with Newton-Raphson method}
       \end{subfigure}
       \begin{subfigure}[b]{0.45\textwidth}
           \includegraphics[width=\textwidth]{graphs/lookbackdeltalrtime.png}
           \caption{Delta with Rayleigh method}
       \end{subfigure}

       \begin{subfigure}[b]{0.45\textwidth}
           \includegraphics[width=\textwidth]{graphs/lookbackoldgammapwlrtime.png}
           \caption{Gamma with Newton-Raphson method}
       \end{subfigure}
       \begin{subfigure}[b]{0.45\textwidth}
           \includegraphics[width=\textwidth]{graphs/lookbackgammapwlrtime.png}
          \caption{Gamma with Rayleigh method}
      \end{subfigure}

        \caption{\label{fig:lboptiongraphs}A graphical analysis of the error of the Greeks depending on the computation time. We notice that the method with Rayleigh distribution is between 10 and 100 times faster. In order to generate these graphs, we have done 1,000 simulations for each value.}
 \end{figure}
 \FloatBarrier


\begin{thebibliography}{9}

\bibitem{lectures}
Harry Zheng (Pr.),
  \textit{Simulation Methods for Finance},
  Imperial College London, London,
  2018.

\bibitem{barrieropt}
Diogo Monteiro da Costa Soares Justino,
  \textit{Hedging of Barrier Options},
  Instituto Universit\'ario de Lisboa, Lisbon,
  2010.

\bibitem{glasserman}
Paul Glasserman (Pr.),
  \textit{Monte Carlo Methods in Financial Engineering},
  Springer Sciencea, New York,
  2013.

\bibitem{exooptions}
Peter G. Zhang (Pr.),
  \textit{Exotic Options, A Guide to Second Generation Options},
  World Scientific Publishing, Hong Kong,
  1998 (second edition).


\end{thebibliography}
\newpage
\pagenumbering{roman}
\part*{Appendix}
\addcontentsline{toc}{part}{Appendix}
\appendix
\section{Sample of closed-formed formula of Barrier Options}
\label{sec:bocomp}
\begin{equation}
\begin{aligned}
UOC(S_0,K,B) ={} &\mathds{1}_{B>K} \left(C_{bs}(S_0, K)-C_{bs}(S_0,B)-(B-K)e^{-rT}\Phi[d_{1}(S_0,B)] \\
      & -\frac{B}{S_0}^{\frac{2v^2}{\sigma^2}}\left[C_{bs}(\frac{B^2}{S_0}, K)-C_{bs}(\frac{B^2}{S_0}, B) -(B_0-K)e^{-rT}\Phi[d_{1}(S_0,B)]\right]\right)
\end{aligned}
\end{equation}
Where $C_{bs}$ and $d_{1}$ are as stated in the Black-Scholes formula (1) and (2), and $v=r-\frac{\sigma^2}{2}$.\\

closed form for DOC price is
$$C_{DO}(S_0,K,B) = C_{bs}(S_0, K) -  (\frac{S_0}{B})^{ -2\frac{\upsilon}{\sigma^2}}C_{bs}(\frac{B^2}{S_0}, K)$$

where $\upsilon = r - {\sigma^2 \over 2}$.

\begin{align*}
\delta_DOC =& \delta_{BS}(S_0, K) - \delta_{BS}(S_0, B)-{B - K\over \sigma S_0 \sqrt{T}}e^{-rT}\Phi(d_2(S_0,B))\\
&+ \left({2\upsilon\over \sigma^2 S_0} \left({B \over S_0}\right)^{ 2\upsilon / \sigma^2} \right)\\
&\times \left(C_{BS}\left({B^2 \over S_0}, K\right) - C_{BS}\left({B^2\over S_0}, B\right) - (B - K)e^{-rT}\Phi(d_2(B, S_0))\right)\\
&-\left({B \over S_0}\right)^{ 2\upsilon / \sigma^2}(\left({-B \over S_0}\right)^2\delta_{BS}\left({B^2 \over S_0}, K\right) + \left({B\over S_0}\right)^2\delta_{BS}\left({B^2 \over S_0}, B\right)\\
&+ \left({B - K \over \sigma S_0\sqrt{T}}\right)e^{-rT}\Phi(d_2(B, S_0)));
\end{align*}

\begin{align*}
\delta_{DOC} =& \Phi\left({\log{S_0 \over K} + (r + {\sigma^2 \over 2})T \over \sigma\sqrt{T}}\right)- \left({B\over S_0}\right)^{r /\sigma^2 - 1}\\
&\times \left(-{B\over S_0}^2 \Phi\left({log{B^2 \over S_0K} + \upsilon T \over \sigma \sqrt{T}} + \sigma \sqrt{T}\right) - {2 \upsilon C_{BS}(B^2 / S_0, K) \over (S_0\sigma^2)}\right)
\end{align*}

\begin{align*}
\gamma_{DOC} =& {\phi(d_2) \over S_0\sigma\sqrt{T}}- \left({B \over S_0}\right)^{ 2\upsilon / \sigma^2}\left({4\upsilon^2 +2\upsilon \sigma^2\over S_0^2\sigma^4}C_{BS}(B^2 / S_0, K) + \gamma_{bs} - {4\upsilon\delta_{bs} \over S_0\sigma^2}\right)
\end{align*}

\begin{align*}
\nu_{DOC} = S_0\phi\left(\frac{log({S_0 \over K}) + (r + {\sigma^2 \over 2})T}{\sigma\sqrt{T}}\right)\sqrt{T}-\left({B \over S_0}\right)^{2\upsilon \over \sigma^2}\left(\nu_{bs} - {4rC_{bs}({B^2\over S_0}, K)log({B\over S_0}) \over \sigma^3}\right)
\end{align*}

\newpage
\section{Code}
All our code is available on our GitHub repository:\\ \url{github.com/tjespel/barrier-and-look-back-options}.

\newpage
\section{Figures}

\begin{figure}[h!]
  \centering
      \begin{subfigure}[b]{0.45\textwidth}
          \includegraphics[width=\textwidth]{graphs/eurodeltalr.png}
          \caption{Delta with LR method}
      \end{subfigure}
      \begin{subfigure}[b]{0.45\textwidth}
          \includegraphics[width=\textwidth]{graphs/eurodeltapw.png}
          \caption{Delta with PW method}
      \end{subfigure}

      \begin{subfigure}[b]{0.3\textwidth}
          \includegraphics[width=\textwidth]{graphs/eurogammalrlr.png}
          \caption{Gamma with LR-LR method}
      \end{subfigure}
      \begin{subfigure}[b]{0.3\textwidth}
          \includegraphics[width=\textwidth]{graphs/eurogammalrpw.png}
          \caption{Gamma with LR-PW method}
      \end{subfigure}
      \begin{subfigure}[b]{0.3\textwidth}
          \includegraphics[width=\textwidth]{graphs/eurogammapwlr.png}
          \caption{Gamma with PW-LR method}
      \end{subfigure}

      \begin{subfigure}[b]{0.45\textwidth}
          \includegraphics[width=\textwidth]{graphs/eurovegalr.png}
          \caption{Vega with LR method}
      \end{subfigure}
      \begin{subfigure}[b]{0.45\textwidth}
          \includegraphics[width=\textwidth]{graphs/eurovegapw.png}
          \caption{Vega with PW method}
      \end{subfigure}

      \caption{\textbf{European Call Option}. A graphical analysis of the error of the Greeks depending on the number of simulations. In order to generate these graphs, we have done 1,000 simulations for each value.}
\end{figure}

\begin{figure}[h!]
  \centering
      \begin{subfigure}[b]{0.45\textwidth}
          \includegraphics[width=\textwidth]{graphs/eurodeltalrtime.png}
          \caption{Delta with LR method}
      \end{subfigure}
      \begin{subfigure}[b]{0.45\textwidth}
          \includegraphics[width=\textwidth]{graphs/eurodeltapwtime.png}
          \caption{Delta with PW method}
      \end{subfigure}

      \begin{subfigure}[b]{0.3\textwidth}
          \includegraphics[width=\textwidth]{graphs/eurogammalrlrtime.png}
          \caption{Gamma with LR-LR method}
      \end{subfigure}
      \begin{subfigure}[b]{0.3\textwidth}
          \includegraphics[width=\textwidth]{graphs/eurogammalrpwtime.png}
          \caption{Gamma with LR-PW method}
      \end{subfigure}
      \begin{subfigure}[b]{0.3\textwidth}
          \includegraphics[width=\textwidth]{graphs/eurogammapwlrtime.png}
          \caption{Gamma with PW-LR method}
      \end{subfigure}

      \begin{subfigure}[b]{0.45\textwidth}
          \includegraphics[width=\textwidth]{graphs/eurovegalrtime.png}
          \caption{Vega with LR method}
      \end{subfigure}
      \begin{subfigure}[b]{0.45\textwidth}
          \includegraphics[width=\textwidth]{graphs/eurovegapwtime.png}
          \caption{Vega with PW method}
      \end{subfigure}

      \caption{\textbf{European Call Option - Time}. A graphical analysis of the error of the Greeks depending on the number of simulations. In order to generate these graphs, we have done 1,000 simulations for each value.}
\end{figure}



\begin{figure}[h!]
  \centering
      \begin{subfigure}[b]{0.45\textwidth}
          \includegraphics[width=\textwidth]{graphs/barrierolddeltalr.png}
          \caption{Delta with LR method}
      \end{subfigure}
      \begin{subfigure}[b]{0.45\textwidth}
          \includegraphics[width=\textwidth]{graphs/barrierdeltalr.png}
          \caption{Delta with LR Rayleigh method}
      \end{subfigure}

      \begin{subfigure}[b]{0.45\textwidth}
          \includegraphics[width=\textwidth]{graphs/barrieroldgammalrlr.png}
          \caption{Gamma with LR-LR method}
      \end{subfigure}
      \begin{subfigure}[b]{0.45\textwidth}
          \includegraphics[width=\textwidth]{graphs/barriergammalrlr.png}
          \caption{Gamma with LR-LR Rayleigh method}
      \end{subfigure}

      \begin{subfigure}[b]{0.45\textwidth}
          \includegraphics[width=\textwidth]{graphs/barrieroldvegalr.png}
          \caption{Vega with LR method}
      \end{subfigure}
      \begin{subfigure}[b]{0.45\textwidth}
          \includegraphics[width=\textwidth]{graphs/barriervegalr.png}
          \caption{Vega with LR Rayleigh method}
      \end{subfigure}

      \caption{\textbf{Barrier Option}. A graphical analysis of the error of the Greeks depending on the number of simulations. In order to generate these graphs, we have done 1,000 simlations for each value, at every power of 10.}
\end{figure}



\begin{figure}[h!]
  \centering
      \begin{subfigure}[b]{0.45\textwidth}
          \includegraphics[width=\textwidth]{graphs/barrierolddeltalr.png}
          \caption{Delta with LR}
      \end{subfigure}
      \begin{subfigure}[b]{0.45\textwidth}
          \includegraphics[width=\textwidth]{graphs/barrierdeltalr.png}
          \caption{Delta with LR Rayleigh}
      \end{subfigure}

      \begin{subfigure}[b]{0.45\textwidth}
          \includegraphics[width=\textwidth]{graphs/barrieroldgammalrlr.png}
          \caption{Gamma with LR-LR method}
      \end{subfigure}
      \begin{subfigure}[b]{0.45\textwidth}
          \includegraphics[width=\textwidth]{graphs/barriergammalrlr.png}
          \caption{Gamma with LR-LR Rayleigh method}
      \end{subfigure}

      \begin{subfigure}[b]{0.45\textwidth}
          \includegraphics[width=\textwidth]{graphs/barrieroldvegalr.png}
          \caption{Vega with LR method}
      \end{subfigure}
      \begin{subfigure}[b]{0.45\textwidth}
          \includegraphics[width=\textwidth]{graphs/barriervegalr.png}
          \caption{Vega with LR Rayleigh method}
      \end{subfigure}

      \caption{\textbf{Look-back Option}. A graphical analysis of the error of the Greeks depending on the number of simulations. In order to generate these graphs, we have done 1,000 simlations for each value, at every power of 10.}
\end{figure}



\begin{figure}[h!]
  \centering
      \begin{subfigure}[b]{0.45\textwidth}
          \includegraphics[width=\textwidth]{graphs/barrierolddeltalrtime.png}
          \caption{Delta with LR}
      \end{subfigure}
      \begin{subfigure}[b]{0.45\textwidth}
          \includegraphics[width=\textwidth]{graphs/barrierdeltalrtime.png}
          \caption{Delta with LR Rayleigh}
      \end{subfigure}

      \begin{subfigure}[b]{0.45\textwidth}
          \includegraphics[width=\textwidth]{graphs/barrieroldgammalrlrtime.png}
          \caption{Gamma with LR-LR method}
      \end{subfigure}
      \begin{subfigure}[b]{0.45\textwidth}
          \includegraphics[width=\textwidth]{graphs/barriergammalrlrtime.png}
          \caption{Gamma with LR-LR Rayleigh method}
      \end{subfigure}

      \begin{subfigure}[b]{0.45\textwidth}
          \includegraphics[width=\textwidth]{graphs/barrieroldvegalrtime.png}
          \caption{Vega with LR method}
      \end{subfigure}
      \begin{subfigure}[b]{0.45\textwidth}
          \includegraphics[width=\textwidth]{graphs/barriervegalrtime.png}
          \caption{Vega with LR Rayleigh method}
      \end{subfigure}


      \caption{\textbf{Look-back Option - Time}. A graphical analysis of the error of the Greeks depending on the number of simulations. In order to generate these graphs, we have done 1,000 simlations for each value, at every power of 10.}
\end{figure}



\begin{figure}[h!]
  \centering
      \begin{subfigure}[b]{0.45\textwidth}
          \includegraphics[width=\textwidth]{graphs/lookbackolddeltalr.png}
          \caption{Delta with LR method}
      \end{subfigure}
      \begin{subfigure}[b]{0.45\textwidth}
          \includegraphics[width=\textwidth]{graphs/lookbackdeltalr.png}
          \caption{Delta with LR Rayleigh}
      \end{subfigure}

      \begin{subfigure}[b]{0.3\textwidth}
          \includegraphics[width=\textwidth]{graphs/lookbackoldgammapwlr.png}
          \caption{Gamma with PW-LR method}
      \end{subfigure}
      \begin{subfigure}[b]{0.3\textwidth}
          \includegraphics[width=\textwidth]{graphs/lookbackgammapwlr.png}
          \caption{Gamma with PW-LR method}
      \end{subfigure}
      \begin{subfigure}[b]{0.3\textwidth}
          \includegraphics[width=\textwidth]{graphs/lookbackgammalrlr.png}
          \caption{Gamma with LR-LR Rayleigh method}
      \end{subfigure}

      \begin{subfigure}[b]{0.45\textwidth}
          \includegraphics[width=\textwidth]{graphs/lookbackvegapw.png}
          \caption{Vega with PW Rayleigh method}
      \end{subfigure}
      \begin{subfigure}[b]{0.45\textwidth}
          \includegraphics[width=\textwidth]{graphs/lookbackvegalr.png}
          \caption{Vega with LR Rayleigh method}
      \end{subfigure}

      \caption{\textbf{LookBack Option}. A graphical analysis of the error of the Greeks depending on the number of simulations. In order to generate these graphs, we have done 1,000 simlations for each value.}
\end{figure}

\begin{figure}[h!]
  \centering
      \begin{subfigure}[b]{0.45\textwidth}
          \includegraphics[width=\textwidth]{graphs/lookbackolddeltalrtime.png}
          \caption{Delta with LR method}
      \end{subfigure}
      \begin{subfigure}[b]{0.45\textwidth}
          \includegraphics[width=\textwidth]{graphs/lookbackdeltalrtime.png}
          \caption{Delta with LR Rayleigh}
      \end{subfigure}

      \begin{subfigure}[b]{0.3\textwidth}
          \includegraphics[width=\textwidth]{graphs/lookbackoldgammapwlrtime.png}
          \caption{Gamma with PW-LR method}
      \end{subfigure}
      \begin{subfigure}[b]{0.3\textwidth}
          \includegraphics[width=\textwidth]{graphs/lookbackgammapwlrtime.png}
          \caption{Gamma with PW-LR method}
      \end{subfigure}
      \begin{subfigure}[b]{0.3\textwidth}
          \includegraphics[width=\textwidth]{graphs/lookbackgammalrlrtime.png}
          \caption{Gamma with LR-LR Rayleigh method}
      \end{subfigure}

      \begin{subfigure}[b]{0.45\textwidth}
          \includegraphics[width=\textwidth]{graphs/lookbackvegapwtime.png}
          \caption{Vega with PW Rayleigh method}
      \end{subfigure}
      \begin{subfigure}[b]{0.45\textwidth}
          \includegraphics[width=\textwidth]{graphs/lookbackvegalrtime.png}
          \caption{Vega with LR Rayleigh method}
      \end{subfigure}

      \caption{\textbf{LookBack Option-Time}. A graphical analysis of the error of the Greeks depending on the number of simulations. In order to generate these graphs, we have done 1,000 simlations for each value.}
\end{figure}




\FloatBarrier

\newpage
\section{App User Guide}
\includepdf[pages=-]{../app/README.pdf}

%\newpage
%\section{Package Manual}
%\includepdf[pages=-]{../src/README.pdf}



\lstset{ %
  language=R,                     % the language of the code
  basicstyle=\footnotesize,       % the size of the fonts that are used for the code
  numbers=left,                   % where to put the line-numbers
  numberstyle=\tiny\color{gray},  % the style that is used for the line-numbers
  stepnumber=1,                   % the step between two line-numbers. If it's 1, each line
                                  % will be numbered
  numbersep=5pt,                  % how far the line-numbers are from the code
  backgroundcolor=\color{white},  % choose the background color. You must add \usepackage{color}
  showspaces=false,               % show spaces adding particular underscores
  showstringspaces=false,         % underline spaces within strings
  showtabs=false,                 % show tabs within strings adding particular underscores
  frame=single,                   % adds a frame around the code
  rulecolor=\color{black},        % if not set, the frame-color may be changed on line-breaks within not-black text (e.g. commens (green here))
  tabsize=2,                      % sets default tabsize to 2 spaces
  captionpos=b,                   % sets the caption-position to bottom
  breaklines=true,                % sets automatic line breaking
  breakatwhitespace=false,        % sets if automatic breaks should only happen at whitespace
  title=\lstname,                 % show the filename of files included with \lstinputlisting;
                                  % also try caption instead of title
  keywordstyle=\color{blue},      % keyword style
  commentstyle=\color{dkgreen},   % comment style
  stringstyle=\color{mauve},      % string literal style
  escapeinside={\%*}{*)},         % if you want to add a comment within your code
  morekeywords={*,...}            % if you want to add more keywords to the set
}
% \lstinputlisting{partA.R}
% \lstinputlisting{partB.R}

\newpage


\end{document}
